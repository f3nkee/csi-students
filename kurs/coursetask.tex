\documentclass[fleqn, a4paper, 12pt, russian]{article}

\usepackage[utf8]{inputenc}
\usepackage[T1, T2A]{fontenc}
\usepackage[english, main = russian]{babel}
\usepackage[style=russian]{csquotes}
\usepackage{pscyr}
\usepackage[top=2cm, left=2cm, right=2cm, left=2cm]{geometry}
\usepackage{amsmath} %Для работы с матрицами

\newcommand\tline[2]{$\underset{\text{#1}}{\text{\underline{\hspace{#2}}}}$}
\newcommand\nameLine[3]{$\underset{\text{#1}}{\text{\underline{\text{#2}\hspace{#3}}}}$}
\pagestyle{empty}

\begin{document}
	\centering
	{\fontsize{14pt}{5cm}\selectfont \bfseries Министерство образования и науки Российской Федерации} \\ \vspace{0.5cm}
	{\fontsize{6.8pt}{5cm}\selectfont ФЕДЕРАЛЬНОЕ ГОСУДАРСТВЕННОЕ АВТОНОМНОЕ ОБРАЗОВАТЕЛЬНОЕ УЧРЕЖДЕНИЕ ВЫСШЕГО ПРОФЕССИОНАЛЬНОГО ОБРАЗОВАНИЯ} \\ 
	\vspace{0.5cm}
	{\fontsize{13pt}{5cm}\selectfont \bfseries “САНКТ-ПЕТЕРБУРГСКИЙ НАЦИОНАЛЬНЫЙ ИССЛЕДОВАТЕЛЬСКИЙ УНИВЕРСИТЕТ ИНФОРМАЦИОННЫХ ТЕХНОЛОГИЙ,} \\ \vspace{0.1cm}
	{\fontsize{13pt}{5cm}\selectfont \bfseries МЕХАНИКИ И ОПТИКИ”} \\ \vspace{1cm}
	
	\flushleft
	{\fontsize{12pt}{0cm}\selectfont {\bfseries КАФЕДРА \underline{\hspace{0.25cm}\textit{Систем Управления и Информатики}\hspace{0.25cm}}} \hspace{1.6cm}УТВЕРЖДАЮ} \\ \vspace{0.1cm}
	{\fontsize{12pt}{0cm}\selectfont \hspace{10.5cm}Зав. кафедрой\hspace{0.6cm}Бобцов А.А.} \\ \vspace{1cm}
	
	\centering
	{\fontsize{18pt}{0cm}\selectfont \bfseries З А Д А Н И Е\hspace{0.2cm}№ 71}\\ \vspace{0.2cm}
	{\fontsize{16pt}{0cm}\selectfont \bfseries на курсовую работу по дисциплине}	\\ \vspace{0.2cm}
	{\fontsize{16pt}{0cm}\selectfont \bfseries \enquote{Теория автоматического управления}} \\ \vspace{1cm}
	
	\flushleft
	{\fontsize{14pt}{0cm}\selectfont \textit{Студенту} \bfseries\hspace{1cm}\underline{\hspace{5.25cm}Поляшову Максиму\hspace{9.20cm}}} \\ \vspace{0.5cm}
	{\fontsize{14pt}{0cm}\selectfont \bfseries \underline{РУКОВОДИТЕЛЬ \hspace{3.17cm} Григорьев Валерий Владимирович\hspace{1cm}}}\\ \vspace{0.3cm}
	\underline{\hspace{\textwidth}} \\ \vspace{0.3cm}
	{\fontsize{14pt}{0cm}\selectfont 1\hspace{0.5cm}Тема проекта \hspace{1cm}\bfseries Синтез регулятора методом построения\\ \underline{\hspace{4.8cm}желаемой ЛАЧХ\hspace{8.25cm}}} \\ \vspace{0.3cm}
	%%
	{\fontsize{14pt}{0cm}\selectfont 2\hspace{0.5cm}Техническое задание: спроектировать регулятор, включённый последовательно с неизменяемой частью системы. Исходные данные\\ \underline{для проектирования:\hspace{12.4cm}}} \\ \vspace{0.2cm}
	{\fontsize{14pt}{0cm}\selectfont \underline{Вид неизменяемой части системы \hspace{3cm}$W(s) = \displaystyle{\frac{K}{(T_1^s+1)(T_2s+1)s}}$\hspace{0.2cm}}} \\ \vspace{0.2cm}
	{\fontsize{14pt}{0cm}\selectfont \underline{Коэффициент передачи неизменяемой части\hspace{5.8cm}170\hspace{0.7cm}}} \\ \vspace{0.2cm}
	{\fontsize{14pt}{0cm}\selectfont \underline{Постоянная времени $T_1$\hspace{10.1cm}0.035 c\hspace{0.4cm}}} \\ \vspace{0.2cm}
	{\fontsize{14pt}{0cm}\selectfont \underline{Постоянная времени $T_2$\hspace{10.1cm}0.35 c\hspace{0.4cm}}} \\ \vspace{0.2cm}
	{\fontsize{14pt}{0cm}\selectfont \underline{Время переходного процесса $t_\text{п}$\hspace{8.4cm}0.15 с\hspace{0.6cm}}} \\ \vspace{0.2cm}
	{\fontsize{14pt}{0cm}\selectfont \underline{Перерегулирование $\sigma$\hspace{10.8cm}25 \% \hspace{0.2cm}}} \\ \vspace{0.2cm}
	{\fontsize{14pt}{0cm}\selectfont \underline{Максимально-допустимое значение амплитуды ${g}_{max}$\hspace{4.8cm}6 \hspace{0.2cm}}} \\ \vspace{0.2cm}
	{\fontsize{14pt}{0cm}\selectfont \underline{Частота гармонического сигнала $w_{0}$\hspace{7.3cm}0.6 1/c\hspace{0.4cm}}} \\ \vspace{0.2cm}
	%%
	\newpage
	{\fontsize{14pt}{0cm}\selectfont \underline{3\hspace{0.5cm}Содержание пояснительной записки (перечень, подлежащих разработке\hspace{0.2cm}}\\\underline{\hspace{0.8cm}вопросов)\hspace{14cm}}} \\ \vspace{0.2cm}
	{\fontsize{14pt}{0cm}\selectfont \underline{\hspace{1.8cm}Введение\hspace{13.15cm}}} \\ \vspace{0.2cm}
	{\fontsize{14pt}{0cm}\selectfont \underline{\hspace{0.8cm}3.1\hspace{0.36cm}Анализ устойчивости неизменяемой части системы \hspace{3.65cm}}} \\ \vspace{0.2cm}
	{\fontsize{14pt}{0cm}\selectfont \underline{\hspace{0.8cm}3.2\hspace{0.36cm}Синтез регулятора\hspace{11.07cm}}} \\ \vspace{0.2cm}
	{\fontsize{14pt}{0cm}\selectfont \underline{\hspace{0.8cm}3.3\hspace{0.36cm}Проверочный расчёт\hspace{10.65cm}}} \\ \vspace{0.2cm}
	{\fontsize{14pt}{0cm}\selectfont \underline{\hspace{0.8cm}3.4\hspace{0.36cm}Реализация регулятора\hspace{10.07cm}}} \\ \vspace{0.2cm}
	{\fontsize{14pt}{0cm}\selectfont \underline{\hspace{1.8cm}Заключение \hspace{12.38cm}}} \\ \vspace{0.2cm}
	{\fontsize{14pt}{0cm}\selectfont \underline{\hspace{1.8cm}Список использованных источников \hspace{7.05cm}}} \\ \vspace{0.2cm}
	%%
	{\fontsize{14pt}{0cm}\selectfont \underline{4\hspace{0.5cm}Исходные материалы и пособия к проекту \hspace{6.7cm}}} \\ \vspace{0.2cm}
	{\fontsize{14pt}{0cm}\selectfont \hspace{0.8cm}4.1\hspace{0.36cm}Учебное пособие. Теория автоматического управления --- Изд. 4,\\ \underline{\hspace{1.8cm}СПб \enquote{Профессия}, 2003 г. \hspace{9.2cm}}} \\ \vspace{0.2cm}
	{\fontsize{14pt}{0cm}\selectfont \hspace{0.8cm}4.2\hspace{0.36cm}Учебное пособие. Правила оформления пояснительной записки и\\ \underline{\hspace{1.8cm}конструкторской документации, Университет ИТМО, 2014 г. \hspace{1.5cm}}} \\ \vspace{0.2cm}
	%%
	\underline{\hspace{\textwidth}} \\ \vspace{0.2cm}
	\underline{\hspace{\textwidth}} \\ \vspace{0.2cm}
	\underline{\hspace{\textwidth}} \\ \vspace{0.2cm}
	\underline{\hspace{\textwidth}} \\ \vspace{0.2cm}
	\underline{\hspace{\textwidth}} \\ \vspace{0.2cm}
	\underline{\hspace{\textwidth}} \\ \vspace{0.2cm}
	\underline{\hspace{\textwidth}} \\ \vspace{0.2cm}
	\underline{\hspace{\textwidth}} \\ \vspace{0.2cm}
	\underline{\hspace{\textwidth}} \\ \vspace{0.2cm}
	\underline{\hspace{\textwidth}} \\ \vspace{0.2cm}
	\underline{\hspace{\textwidth}} \\ \vspace{0.2cm}
	\underline{\hspace{\textwidth}} \\ \vspace{0.2cm}
	%%
	\vspace{2cm}
	{\fontsize{14pt}{0cm}\selectfont
		{5 \hspace{0.5cm}Дата выдачи задания \hspace{1.5cm} \underline{\hspace{9.5cm}}} \\ \vspace{0.2cm}
		{\hspace{4.2cm}Руководитель\hspace{0.25cm}\underline{\hspace{9.5cm}}} \\ \vspace{0.2cm}
		{\hspace{0.25cm}Задание принято к исполнению\hspace{0.25cm}\underline{\hspace{9.5cm}}} \\ \vspace{0.2cm}
		{\hspace{3.3cm}Подпись студента\hspace{0.25cm}\underline{\hspace{9.5cm}}}
	}
\end{document}